\documentclass{article}
\usepackage{blindtext}
\usepackage{titlesec}
\title{Documentation}
\author{Jerry Wang}
\date{ }
\usepackage{amsmath}
\usepackage{amsfonts}
\usepackage{mathtools}
\usepackage{graphicx}
\usepackage{geometry}
\geometry{margin=1in}



\usepackage{color}  
\usepackage{hyperref}
\hypersetup{
    colorlinks=false,% hyperlinks will be black
    pdfborderstyle={/S/U/W 1}% border style will be underline of width 1pt
}

\begin{document}

\maketitle

\tableofcontents
\newpage
\section{Introduction}
\subsection{Entropy - Surprise Factor}
First we would like to define the concept of the "surprise" of an event $E$ occurring and define entropy based off that. We deem this important since entropy plays a large role in many of the metric developed and implemented. As such, it would be helpful to define a consistent mathematical framework as well as develop intuition for these definitions at a precise mathematical level. 

The following methods and definitions are based on Ross's A First Course in Probability \cite{b1}. Mathematically, it makes sense that the surprise invoked by an event $E$ occurring should be a function of the probability of event $E$ itself, which we will denote $p$.
Thus we define $S(p)$ as the surprise invoked by an event with a probability of $p$. 
Now we will begin by stating some axioms for this definition of surprise.

\textbf{Axiom 1}
\[S(1) = 0\]
Intuitively, that just means we should feel no surprise when a event with probability $1$ occurs. That is, it is not surprising at all for a sure event to occur.

\textbf{Axiom 2}
\[
p<q \implies S(p)<S(q)
\]
That is, $S(p)$ is a strictly decreasing function of $p$.

\textbf{Axiom 3}
\[
S(p) \text{ is a continuous function with respect to $p$}
\]

\textbf{Axiom 4}
\[
S(pq) =S(p)+S(q)
\]
The intuition of this axiom is given when we consider independent events $E_1$ and $E_2$ with probabilities of occurring with $p$ and $q$ respectively. 
Since $E_1$ and $E_2$ are independent, $P(E_1 \cap E_2) = P(E_1)\cdot P(E_2) = pq$. 
Thus the surprise invoked by $E_1 \cap E_2$ should be defined as $S(pq)$. 
But now, let us consider that we are first told that event $E_1$ occurred, and then afterwards event $E_2$ occur. 
We know the total surprise invoked is simply the surprised invoked by $E_1 \cap E_2$ which is $S(pq)$, and since we know that $S(p)$ is the surprise invoked by event $E_1$ alone, it follows that $S(pq) - S(p)$ should represent the initial surprise invoked by $E_2$. 
Due to independence, we know the probability of $E_2$ is still $q$ and thus the surprise of event $E_2$ should remain $S(q)$. 
Thus we have that $S(pq) - S(p) = S(q)$ or that $S(pq) =S(p)+S(q)$. 

\bigskip

From the axioms, we can prove that
\[
S(p) = -C\log_2 p
\]
where $C > 0$.
From Axiom 4, we have that 
\[
S(p^2)=S(p)+S(p)+2S(p)
\]
From induction, we also have that for $m\in \mathbb{Z} > 0$
\begin{equation} \label{inducResult}
    S(p^m)=mS(p)
\end{equation}

Also note that for $n\in \mathbb{Z} > 0$
\[
S(p) = S\left((p^\frac{1}{n})^n\right) = nS(p^\frac{1}{n})
\]
which implies that
\begin{equation} \label{inverseRes}
    S(p^\frac{1}{n}) = \frac{S(p)}{n}
\end{equation}

Combining equations \ref{inducResult} and \ref{inverseRes}, we can define
\begin{equation} \label{rationalEq}
    S(p^x)=xS(p)
\end{equation}
for $x \in \mathbb{Q} >0$. Now from Axiom 3, we can define equation \ref{rationalEq} $\forall x \in\mathbb{Q} >0$.

Now let us take $x=-\log_2 p$ which implies $p=\left(\frac{1}{2}\right)^x$ which allows for the following relation.
\[
S(p) = S\left((\frac{1}{2})^x\right) = xS\left(\frac{1}{2}\right) = -S\left(\frac{1}{2}\right)\cdot\log_2 p
\]
Now notice $S\left(\frac{1}{2}\right) = C$ is simply a constant (that is non-zero thanks to Axiom 1 and Axiom 2).

Thus we have shown that
\begin{equation}
    S(p) = -C\log_2 p
\end{equation}
follows from the axioms.
Lastly, note that it is standard to let $C=1$.


\section{Mathematical Metrics}
\subsection{Shannon Entropy Metric}

\subsubsection{Shannon Entropy Ratio}

\subsection{Password Entropy Metric}

\newpage
\begin{thebibliography}{00}
\bibitem{b1} Ross, Sheldon M. 2019. A First Course in Probability. 10th ed. Pearson.
\end{thebibliography}

\end{document}