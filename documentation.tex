\documentclass{article}
\usepackage{blindtext}
\usepackage{titlesec}
\title{Weekly Report}
\author{Jerry Wang}
\date{ }
\usepackage{amsmath}
\usepackage{mathtools}
\usepackage{graphicx}
\usepackage{geometry}
\geometry{margin=1in}



\usepackage{color}  
\usepackage{hyperref}
\hypersetup{
    colorlinks=false,% hyperlinks will be black
    pdfborderstyle={/S/U/W 1}% border style will be underline of width 1pt
}

\begin{document}

\maketitle

\tableofcontents
\newpage
\section{Introduction}
\subsection{Entropy - Surprise Factor}
First we would like to define the concept of the "surprise" of an event $E$ occurring. The following methods and definitions are based on Ross's A First Course in Probability \cite{b1}. Mathematically, it makes sense that the surprise invoked by an event $E$ occurring should be a function of the probability of event $E$ itself, which we will denote $p$.
Thus we define $S(p)$ as the surprise invoked by an event with a probability of $p$. 
Now we will begin by stating some axioms for this definition of surprise.

\textbf{Axiom 1}
\[S(1) = 0\]
Intuitively, that just means we should feel no surprise when a event with probability $1$ occurs. That is, it is not surprising at all for a sure event to occur.

\textbf{Axiom 2}
\[
p<q \implies S(p)<S(q)
\]
That is, $S(p)$ is a strictly decreasing function of $p$.

\textbf{Axiom 3}
\[
S(p) \text{ is a continuous function with respect to $p$}
\]

\newpage
\begin{thebibliography}{00}
\bibitem{b1} Ross, Sheldon M. 2019. A First Course in Probability. 10th ed. Pearson.
\end{thebibliography}

\end{document}